\documentclass[12pt]{article}
\usepackage{times}
\title{Pick Up}
\author{Ryan Miura (miurary) and Michael Chan (chanmic)}
\begin{document}
	\maketitle
	\tableofcontents
		\section{Problem Description}
        	\paragraph{Physical activity is necessary to stay healthy. However, only 21\% of adults meet the daily requirement of physical activity\cite{cdc}. Sports provide a fun medium to remain physically active, but most sports require a team. In our busy society, it may be difficult to find time and others to play with, especially for older adults with jobs. Additionally, some people are more competitive than others, making it intimidating to join pick-up games.}
        \section{Solution}
        	\paragraph{The main way people found others to play with in the past was to head to the park and see if there was anyone there playing. This approach can be disappointing and a waste of time if no one is there.}
        	\paragraph{Our solution to this problem is Pick Up, a website that allows for people to connect with others in their community through sports. Users will be asked to create or log into an account. The site will then have a list of various sports the user can select from. Once a sport has been selected, a user can browse posts from other users and find people to play with. They can also make a post themselves, either as a part of a group or as a single person. When making a post, the user must indicate how serious they are, ranging from casual, serious, and hardcore. If a user happens to notice many individual posters, they can also make a group and invite all the individual posters to it to try and form a group that way. Pick Up ensures there is no time wasted in commuting to an empty park. Simply exchange a few messages with someone nearby, meet up, and play.}
            \paragraph{What makes Pick Up so useful is the filtering. With a variable search range, users can adjust it based on how far they are willing to travel. Also, filtering by seriousness allows for more compatible matches between groups and individuals. This way, someone just learning the game can find a casual group instead of people who treat every game like the world championship.}
        \section{Challenges}
            \paragraph{The biggest challenge in developing this product that we see is implementing a log-in system. This system would have to verify a user's log-in information and then update the page to match their preferences. It would also have to store their past messages and load them in. All of this would most likely be done using a database such as MongoDB or something like it. To minimize this risk, someone with experience in MongoDB or log-in systems should be included in the development team. Otherwise, the team should do research together in order to become well-versed in what's required to implement a log-in system.}
            \paragraph{The biggest challenge post-release is to get people to use it. Pick Up's usefulness increases with the amount of people using it. Therefore, informing people of its existence is one challenge. Getting them to continually use it is another. By forming a solid marketing plan and then providing a way for users to give feedback would be a good way to help solve this problem. Interactivity between users and the developers creates memorable experiences and prompts the continued use of the website.}
        \section{Resources}
        		\paragraph{This project can be completed using the OSU servers as we have in the past. HTML, CSS, JavaScript, nodejs, express, and MongoDB would all be used to develop the website. None of these require anything more than a text editor (Puddy, MobaXTerm, etc.) and access to the ENGR servers.}
        
\begin{thebibliography}{9}
\bibitem{cdc}
CDC. Facts About Physical Activity, 
\\\texttt{https://www.cdc.gov/physicalactivity/data/facts.htm}
\end{thebibliography}

\end{document}

